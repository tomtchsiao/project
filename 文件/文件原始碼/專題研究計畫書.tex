\documentclass[a4paper]{article}
\usepackage{fontspec}
\usepackage{ctex}
\setCJKmainfont{標楷體-繁}
\setmainfont{Times New Roman}
\usepackage{amsmath}
\usepackage{array}
\usepackage{geometry}
\usepackage{titlesec}
\usepackage{fancyhdr}
\usepackage{caption}
\usepackage{lastpage}
\geometry{left=3cm, right=3cm, top=2.5cm, bottom=2.5cm}
\titleformat{\section}{\large\bfseries}{\thesection}{1em}{}
\titleformat{\subsection}{\normalsize\bfseries}{\thesubsection}{1em}{}
\pagestyle{fancy}
\fancyhf{}
\fancyhead[L]{國立中正大學專題研究計畫書}
\fancyhead[R]{\thepage\ of \pageref{LastPage}}
\begin{document}
\title{\fontsize{16pt}{16pt}\selectfont 國立中正大學專題研究計畫書}
\author{
    資工三 \hspace{0.1 cm} 蕭子期 411410055, 資工三 \hspace{0.1 cm} 許文碩 411410077 \\ \\
    指導教授: 柯仁松教授
}
\date{}
\maketitle
\fontsize{12pt}{12pt}\selectfont
\begin{center}
專題關鍵字: 健康, 體重計算, 每日飲食, 運動, 網站開發
\end{center}
\section*{壹、專案名稱}
\noindent
The impact of daily diet and exercise on weight. (飲食及運動對體重影響)
\section*{貳、簡述}
\noindent
現代人對健康議題越加重視,了解身體各項數據不僅能提早發現潛在的健康問題,亦能及時調整不健康的作息和飲食習慣。本專題旨在架設一個網站,讓用戶能夠預測每日飲食和運動對體重的影響。透過這個網站,用戶可以輸入他們的日常飲食、運動習慣以及其他生活方式的數據,系統將根據這些數據計算出實際體重變化,並提供圖表顯示體重趨勢。網站還會提供 BMI, TDEE 等健康相關之建議。這個平台不僅是一個追蹤體重的工具,更是一個健康管理的夥伴,幫助用戶做出更明智的健康選擇,改善整體生活品質。透過這個網站,用戶可以更好地了解自己的身體狀況,並採取積極的措施來維持或改善健康。
\section*{參、專題技術}
\noindent
程式語言: Python, HTML, CSS \\
開發工具: Flask \\
美化: JavaScript
\section*{肆、初步規劃}
\noindent
分為前端和後端,前端主要是網頁建立及美化,後端是 Python 計算及前後端互相傳送之 Flask。   \\ \\
一、HTML 架設網站及美化。 \\
二、Python 語言計算建議體重及 Flask。 \\
三、JavaScript 直方圖顯示。
\section*{伍、預期成果}
\noindent
提供用戶一個可視化的體重結果,幫助他們更直觀地了解體重變化和健康狀況。
\end{document}